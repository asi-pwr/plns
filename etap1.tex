\subsection*{Temat}
Zintegrowany system zarządzania rezerwacjami w kinie


\subsection*{Cele}

Zadaniem systemu jest usprawnienie procesu rezerwowania miejsc na spektaklach filmowych. Dzięki systemowi, klienci będą mogli zdalnie dowiadywać się o spektaklach, które odbędą się w kinie i składać na nie rezerwacje. Mogliby dokładnie określić miejsce, które chcą zarezerwować. Tradycyjna forma rezerwacji - w kinie, przez kasjera - ciągle będzie możliwa. Dzięki nowemu systemowi ilość rezerwacji dokonywanych ręcznie przez kasjerów powinna zmaleć - szacuje się, że 60\% klientów będzie składało rezerwację zdalnie. Pozwoli to zredukować ilość pracowników i w efekcie zmniejszyć koszty funkcjonowania kina.

Dzięki systemowi menedżerowie lub kierownicy kina będą mieli wgląd jakiego rodzaju filmy cieszą się największą popularnością. Pozwoli im to skuteczniej proponować kolejne spektakle. Ilość klientów przychodzących do kina (a w związku z tym przychody firmy) wzrośnie.
Dzięki gromadzeniu danych klientów (w tym ich adresy poczty elektronicznej) będzie możliwe rozsyłanie spersonalizowanych wiadomości reklamowych przez zewnętrzny system. W bazie danych będzie zgromadzona historia ich rezerwacji, co pozwoli sprawdzić jakie filmy najbardziej ich interesują - skuteczność reklam będzie wysoka.

Wymienione aspekty przyniosą firmie wymierne korzyści - wzrost przychodów i obniżenie wydatków na personel.


\subsection*{Założenia}

\begin{itemize}
	\item Zarządzanie klientami.
    \item Gromadzenie danych o filmach.
	\item Kontrolowanie dostępności miejsc w salach.
    \item Zarządzenie rezerwacjami.
\end{itemize}