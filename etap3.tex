\kategoria{Sala}
\opis Kategoria ,,Sala'' przechowuje dane sal należących do kina składających się z miejsc.
\begin{atrybuty}
    \item Numer sali -- numer przydzielony danej sali, np. 5
    \item Numer piętra -- piętro na którym jest sala, np. 2
\end{atrybuty}

\kategoria{Miejsce}
\opis Kategoria ,,Miejsce'' przechowuje dane miejsc w salach należących do kina.
\begin{atrybuty}
    \item Numer miejsca -- numer miejsca w danej sali, np. 10
    \item Numer rzędu -- rząd w którym jest miejsce, np. 1
\end{atrybuty}

\kategoria{Spektakl}
\opis Kategoria ,,Spektakl'' przechowuje dane seansu odbywającego się w danej sali, na którym wyświetlany jest film.
\begin{atrybuty}
    \item Data i czas rozpoczęcia -- data i czas rozpoczęcia seansu, np. 2012-04-01 13:30:00
    \item Data i czas zakończenia -- data i czas zakończenia seansu, np. 2012-04-01 15:30:00
\end{atrybuty}

\kategoria{Rezerwacja}
\opis Kategoria ,,Rezerwacja'' przechowuje dane rezerwacji złożonych na dane miejsce i spektakl przez klienta lub kasjera.
\begin{atrybuty}
    \item Data i czas wykonania -- data i czas wykonania rezerwacji, np. 2012-04-01 13:00:00
\end{atrybuty}

\kategoria{Film}
\opis Kategoria ,,Film'' służy do opisu filmu, który może być wyświetlany na wielu seansach, w którego tworzeniu brało udział wiele osób i który może być zakwalifikowany do wielu gatunków.
\begin{atrybuty}
    \item Tytuł -- pełny tytuł filmu, np. Forrest Gump
    \item Opis -- tekst opisowy filmu, np. "Forrest Gump" to romantyczna historia, w której Tom Hanks wcielił się w tytułową postać (...)
    \item Data premiery -- światowa data premiery filmu, np. 1994-06-23
	\item Kraj pochodzenia -- państwo, z którego pochodzi film, np. Polska
    \item Czas trwania -- czas jaki trwa film podany w sekundach, np. 7200
    \item Ograniczenie wieku -- minimalny wiek od którego można oglądać film, np. 18
\end{atrybuty}

\kategoria{Gatunek}
\opis Kategoria ,,Gatunek'' przechowuje gatunki, do których mogą być zakwalifikowane filmy.
\begin{atrybuty}
    \item Nazwa -- nazwa gatunku, np. Komedia
\end{atrybuty}

\kategoria{Komentarz}
\opis Kategoria ,,Komentarz'' przechowuje komentarze dodane przez klientów na dowolne filmy.
\begin{atrybuty}
    \item Tytuł -- tytuł komentarza, np. Polecam ten film!
    \item Zawartość -- treść komentarza, np. Polecam film ze względu na świetne efekty specjalne.
    \item Data i czas napisania -- data i czas napisania komentarza przez klienta, np. 2012-04-03 15:30:00
\end{atrybuty}

\kategoria{Ocena}
\opis Kategoria ,,Ocena'' przechowuje oceny klientów o filmach.
\begin{atrybuty}
    \item Stopień -- wartość oceny, np. 3
    \item Data i czas dodania -- data i czas ocenienia filmu przez klienta, np. 2012-04-03 15:20:00
\end{atrybuty}
	
\kategoria{Klient}
\opis Kategoria ,,Klient'' opisuje klientów w systemie.
\begin{atrybuty}
    \item Nazwa -- unikatowa nazwa użytkownika, np. Ania\_123
    \item Hasło -- hasło wymagane do zalogowania do systemu, np. bardzotajnehaslo
	\item E-mail -- adres poczty elektronicznej, np. ania\_123\@poczta.pl
	\item Imię -- imię klienta, np. Ania
	\item Nazwisko -- nazwisko klienta systemu, np. Kowalska
	\item Data urodzenia -- data urodzenia klienta, np. 1991-02-05
\end{atrybuty}

\kategoria{Filmowiec}
\opis Kategoria ,,Filmowiec'' przechowuje dane osób uczestniczących w tworzeniu filmów, np. aktorów, reżyserów itp.
\begin{atrybuty}
    \item Imię -- imię filmowca, np. Cezary
    \item Nazwisko -- nazwisko filmowca, np. Pazura
    \item Data urodzenia -- dzień w którym urodziła się dana osoba, np. 1970-04-01
    \item Narodowość -- kraj z którego pochodzi osoba, np. Polska
\end{atrybuty}
